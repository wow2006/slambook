\section{nonlinear least squares}

\label{sec:6.2}

Let's consider a simple least squares problem first:

\begin{equation}
\mathop {\min }\limits_{\bm{x}} F(\bm{x}) = \frac{1}{2}{\left\| {f\left( \bm{x} \right)} \right\|^2_2}.
\end{equation}

Where the argument $\bm{x} \in \mathbb{R}^n$, $f$ is an arbitrary scalar nonlinear function $f(\bm{x}): \mathbb{R}^n \mapsto \ Mathbb{R}$. Note that the coefficient $\frac{1}{2}$ is irrelevant. Some documents have this coefficient, and some documents do not. It does not affect the subsequent conclusions. Let's discuss how to solve such an optimization problem. Obviously, if $f$ is a mathematically simple function, then the problem can be solved in analytical form. Let the derivative of the objective function be zero, and then solve the optimal value of $\bm{x}$, just like the extreme value of the binary function:

\begin{equation}
\frac{ \mathrm{d} F}{ \mathrm{d} \bm{x} } = \bm{0}.
\end{equation}

Solving this equation yields an extremum with a derivative of zero. They may be maximal, very small, or values at the saddle point, as long as they compare the size of their function values one by on. But is this equation easy to solve? This depends on the form of the $f$ derivative. If $f$ is a simple linear function, then the problem is a simple linear least squares problem, but some derivatives may be complex in form, making the equation not easy to solve. Solving this equation requires us to know the \textbf{global nature} of the objective function, which is usually not possible. For the least squares problem that is inconvenient to solve directly, we can use the \textbf{iteration} method to continuously update the current optimization variable from an initial value to make the objective function drop. The specific steps can be listed as follows:

\begin{mdframed}  
	\begin{enumerate}
		\item gives an initial value of $\bm{x}_0$.
		\item For the $k$ iteration, look for an increment of $\Delta \bm{x}_k$, making $\left\| {f\left( \bm{x}_k + \Delta \bm{x} _k \right)} \right \|^2_2$ reaches a minimum value.
		\item Stop if $\Delta \bm{x}_k$ is small enough.
		\item Otherwise, let $\bm{x}_{k+1} = \bm{x}_k+\Delta \bm{x}_k$ return to step 2.
	\end{enumerate}
\end{mdframed}

This turns the problem of solving the \textbf{The derivative function is zero} The problem has become a constant \textbf{find drop increment} $\Delta \bm{x}_k$ problem, as we will see, since it can be $f$ Linearization, the calculation of the increment will be much simpler. When the function drops until the increment is very small, the algorithm is considered to converge and the objective function reaches a minimum value. In this process, the problem is how to find the increment of each iteration point, and this is a local problem, we only need to care about the local nature of $f$ at the iteration value rather than the global nature. Such methods are widely used in areas such as optimization and machine learning.
	
Next we look at how to find this increment $\Delta \bm{x}_k$. This part of the knowledge is actually in the field of numerical optimization, let's look at some widely used results.

\subsection{BCH formula and approximation}

A major motivation for using Lie algebra is to optimize, and the derivative is a very necessary information in the optimization process (we will cover it in detail in Lecture 6). Let's consider a problem below. Although we have already understood the relationship between Lie group and Lie algebra on $\mathrm{SO}(3)$ and $\mathrm{SE}(3)$, but in $\mathrm{SO}(3)$ What happens to $\mathfrak{so}(3)$ in Lie algebra when two matrix multiplications are completed? Conversely, when $\mathfrak{so}(3)$ is used to add two Lie algebras, does $\mathrm{SO}(3)$ correspond to the product of the two matrices? If established, it is equivalent to:

\[
  \exp \left( {\boldsymbol{\phi} _1^ \wedge } \right)\exp \left( {\boldsymbol{\phi} _2^ \wedge } \right) = \exp \left( {{{\left( {{\boldsymbol{\phi} _1} + {\boldsymbol{\phi} _2}} \right)}^ \wedge }} \right) ?
\]

If $\boldsymbol{\phi}_1, \boldsymbol{\phi}_2$ is a scalar, then obviously this is true; but here we calculate the exponential function of \textbf{matrix} instead of the scalar exponent. In other words, we are studying whether the following formula holds:

\[
  \ln \left( \exp \left( \bm{A} \right) \exp \left( \bm{B} \right) \right) = \bm{A} + \bm{B} \; ?
\]

Unfortunately, this formula does not hold true in the matrix. The complete form of the product of the two Lie algebra indices is represented by the Baker-Campbell-Hausdorff formula (BCH formula)\footnote{ see \url{https://en.wikipedia.org/wiki/Baker-Campbell-Hausdorff\_formula}. } given. Due to the complexity of its complete form, we only give the first few items of its expansion:

\begin{equation}
  \ln \left( {\exp \left( \bm{A} \right)\exp \left( \bm{B} \right)} \right) = \bm{A} + \bm{B} + \frac{1}{2}\left[ {\bm{A}, \bm{B}} \right] + \frac{1}{{12}}\left[ {\bm{A},\left[ {\bm{A}, \bm{B}} \right]} \right] - \frac{1}{{12}}\left[ {\bm{B},\left[ {\bm{A},\bm{B}} \right]} \right] +  \cdots
\end{equation}

Where $[]$ is the brackets. The BCH formula tells us that when dealing with the product of two matrix indices, they produce some remainders consisting of Lie brackets. In particular, consider the Lie algebra on $\mathrm{SO}(3)$$\ln { \left( {\exp \left( { \boldsymbol{\phi} _1^ \wedge } \right)\exp \left ( {\boldsymbol{\phi} _2^ \wedge } \right)} \right) ^ \vee }$, when $\boldsymbol{\phi_1}$ or $\boldsymbol{\phi_2}$ is small, small Items with more than two quantities can be ignored. At this time, BCH has a linear approximation \footnote{ concrete derivation of BCH specific form and approximate expression, this book is not discussed, please refer to the document \cite{Barfoot2016}. }:

\begin{equation}
  \ln { \left( {\exp \left( { \boldsymbol{\phi} _1^ \wedge } \right)\exp \left( {\boldsymbol{\phi} _2^ \wedge } \right)} \right) ^ \vee } \approx \left\{
    \begin{array}{l}
      {\bm{J}_l}{\left( {{\boldsymbol{\phi} _2}} \right)^{ - 1}}{ \boldsymbol{\phi} _1} + {\boldsymbol{\phi} _2} \quad \text{when} \boldsymbol{\phi}_1 \text{Small amount},\\
      {\bm{J}_r}{\left( {{\boldsymbol{\phi} _1}} \right)^{ - 1}}{\boldsymbol{\phi} _2} + {\boldsymbol{\phi} _1} \quad \text{when} \boldsymbol{\phi}_2 \text{Small amount}.
  \end{array} \right.
\end{equation}

Take the first approximation as an example. This formula tells us to multiply a tiny rotation matrix $\bm{R}_1$ to a rotation matrix $\bm{R}_2$ (the Lie algebra is $\boldsymbol{\phi}_2$) (Li algebra is $\boldsymbol{\phi} _1$) can be approximated as the original Lie algebra $\boldsymbol{\phi}_2$Added an item${\bm{J}_l}{\left( {{\boldsymbol{\phi} _2}} \right)^{ - 1}}{ \boldsymbol{\phi} _1}$. 
Similarly, the second approximation describes the case where the right multiplied by a small displacement. Therefore, under the BCH approximation, the Lie algebra is divided into a left-multiplying approximation and a right-multiplying approximation. In use, we must pay attention to whether the left-multiplier model or the right-multiply model is used.

This book takes the left multiplication as an example. Left multiply BCH approximates Jacobian $\bm{J}_l$ is actually the content of the form ~\eqref{eq:lieAlgebraJacobian}~:

\begin{equation} 
{ \bm{J}_l} = \bm{J} = \frac{{\sin \theta }}{\theta } \bm{I} + \left( {1 - \frac{{\sin \theta }}{\theta }} \right) \bm{a} { \bm{a}^\mathrm{T}} + \frac{{1 - \cos \theta }}{\theta }{ \bm{a}^ \wedge}.
\end{equation}

Its inverse is:

\begin{equation}
\bm{J}_l^{ - 1} = \frac{\theta }{2}\cot \frac{\theta }{2} \bm{I} + \left( {1 - \frac{\theta } {2}\cot \frac{\theta }{2}} \right) \bm{a} {\bm{a}^\mathrm{T}} - \frac{\theta }{2}{ \bm{ a}^ \wedge }.
\end{equation}

Right-handed Jacobi only needs to take a negative sign for the argument:

\begin{equation}
\bm{J}_r(\boldsymbol{\phi}) =\bm{J}_l(-\boldsymbol{\phi}) .
\end{equation}

In this way, we can talk about the relationship between Lie group multiplication and Lie algebra addition.

For the convenience of the reader, we restate the meaning of the BCH approximation. Suppose that for a rotation of $\bm{R}$, the corresponding Lie algebra is $\boldsymbol{\phi}$. We give it a small rotation to the left, denoted as $\Delta \bm{R}$, and the corresponding Lie algebra is $\Delta \boldsymbol{\phi}$. Then, on Lie group, the result is $ \Delta \bm{R} \cdot \bm{R}$, and on the Lie algebra, according to the BCH approximation, it is $\bm{J}_l^{-1 } (\boldsymbol{\phi}) \Delta \boldsymbol{\phi} + \boldsymbol{\phi}$. Combined, you can simply write:

\begin{equation}
\exp \left( {\Delta { \boldsymbol{\phi} ^ \wedge }} \right)\exp \left( {{ \boldsymbol{\phi} ^ \wedge }} \right) = \exp \left( {{{\left( { \boldsymbol{\phi} + \bm{J}_l^{ - 1}\left( \boldsymbol{\phi} \right)\Delta \boldsymbol{\phi} } \right)} ^ \wedge }} \right).
\end{equation}

Conversely, if we add on Lie algebra and add $\boldsymbol{\phi}$ to $\Delta \boldsymbol{\phi}$, we can approximate the multiplication of the left and right Jacobian on the Lie group:

\begin{equation}
\exp \left( {{{\left( { \boldsymbol{\phi}  + \Delta \boldsymbol{\phi} } \right)}^ \wedge }} \right) = \exp \left( {{{\left( {{ \bm{J}_l}\Delta \boldsymbol{\phi} } \right)}^ \wedge }} \right)\exp \left( {{ \boldsymbol{\phi} ^ \wedge }} \right) = \exp \left( {{\boldsymbol{\phi} ^ \wedge }} \right)\exp \left( {{{\left( {{\bm{J}_r}\Delta \boldsymbol{\phi} } \right)}^ \wedge }} \right).
\end{equation}

This provides a theoretical basis for calculus after Lie algebra. Similarly, for $\mathrm{SE}(3)$, there is a similar BCH approximation:

\begin{equation}
\exp \left( {\Delta {\boldsymbol{\xi} ^ \wedge }} \right)\exp \left( {{ \boldsymbol{\xi} ^ \wedge }} \right) \approx \exp \left ( {{{\left( {{ \bm{\mathcal{J}}_l^{-1} }\Delta \boldsymbol{\xi} + \boldsymbol{\xi} } \right)}^ \wedge }} \right),
\end{equation}

\begin{equation}
\exp \left( {{ \boldsymbol{\xi} ^ \wedge }} \right) \exp \left( {\Delta {\boldsymbol{\xi} ^ \wedge }} \right) \approx \exp \left ( {{{\left( {{ \bm{\mathcal{J}}_r^{-1} }\Delta \boldsymbol{\xi} + \boldsymbol{\xi} } \right)}^ \wedge }} \right).
\end{equation}

Here the $\bm{\mathcal{J}}_l$ form is more complicated. It is a matrix of $6 \times 6$. Readers can refer to the contents of the \cite{Barfoot2016} formulas (7.82) and (7.83). Since we did not use the Jacobi in the calculation, the actual form is omitted here.


\subsubsection{Install OpenCV}

OpenCV\footnote{Official homepage:\url{http://opencv.org}. } provides a large number of open source image algorithms, is a library of image processing algorithms used in computer vision. This book also uses OpenCV for basic image processing. Before using it, it is recommended that the reader install it from the source code. Under Ubuntu, there are two ways to \textbf{install from source code} and \textbf{install library file only}:

\begin{enumerate}
	\item Installation from source code means downloading all OpenCV source code from the OpenCV website and compiling and installing it on the machine for use. The advantage is that the version you can choose is rich, and you can see the source code, but it takes some compilation time.
	\item only installs the library file, which means to install the library file compiled by Ubuntu community personnel through Ubuntu, so there is no need to recompile it.
\end{enumerate}

Since we are using a newer version of OpenCV, we must install it from the source code. As a result, you can adjust some compile options to match the programming environment (for example, GPU acceleration is not required); in addition, source code installation can use some additional features. OpenCV currently maintains two major versions, divided into OpenCV 2.4 series and OpenCV 3 series. This book uses the OpenCV \textbf{3} series.

Since the OpenCV project is relatively large, it is not placed under the 3rdparty of this book. Readers can download from ~\url{http://opencv.org/downloads.html} and choose OpenCV for Linux. You will get a tarball like opencv-3.1.0.zip. Extract it to any directory, we found that OpenCV is also a cmake project.

Before compiling, first install the OpenCV dependencies:

\begin{lstlisting}[language=sh,caption=terminal input:]
Sudo apt-get install build-essential libgtk2.0-dev libvtk5-dev libjpeg-dev libtiff4-dev libjasper-dev libopenexr-dev libtbb-dev
\end{lstlisting}

In fact, OpenCV has a lot of dependencies, and the lack of some compiled items will affect some of its features (but we won't use all the features). OpenCV will check if the dependencies will be installed and adjust its functionality during the cmake phase. If you have a GPU on your computer and have related dependencies installed, OpenCV will also speed up the GPU. But for this book, the above dependencies are enough.

Subsequent compilation and installation is the same as the normal cmake project. After make, call sudo make install to install OpenCV on your machine (instead of just compiling it). Depending on the machine configuration, this compilation process can take anywhere from twenty minutes to an hour. If your CPU is strong, you can use a command like "make -j4" to call multiple threads to compile (the arguments after -j are the number of threads used). After installation, OpenCV is stored by default in the /usr/local directory. You can find the location where OpenCV header files and library files are installed and see where they are. In addition, if you have previously installed the OpenCV 2 series, it is recommended that you install OpenCV 3 somewhere else (think how this should work).


\subsection{Lie algebra}

First, consider the situation on $\mathrm{SO}(3)$. Suppose we rotate a space point $\bm{p}$ and get $\bm{R} \bm{p}$. Now, to calculate the coordinates of the point after rotation relative to the derivative of the rotation, we informally write it as \footnote{Please note that the derivative cannot be defined by matrix differentiation, which is just a token. }:

\[
\frac{{\partial \left( {\bm{Rp}} \right)}}{{\partial \bm{R}}}.
\]

Since $\mathrm{SO}(3)$ has no addition, the derivative cannot be calculated as defined by the derivative. Let the Lie algebra corresponding to $\bm{R}$ be $\boldsymbol{\phi}$, and we will calculate \footnote{ strictly speaking. In matrix differentiation, we can only find the derivative of the row vector about the column vector. The result is a matrix. However, this book writes the derivative of the column vector to the column vector. The reader can think that the molecule is transposed first, and then the final result is transposed. This makes the formula simple, otherwise we have to add a transpose to each line of the molecule. In this sense, you can think of $\mathrm{d}(\bm{Ax})/\mathrm{d}\bm{x} = \bm{A}$. }:

\[ \frac{{\partial \left( {\exp \left( \boldsymbol{\phi} ^ \wedge \right) \bm{p}} \right)}}{{\partial \boldsymbol{\phi} }}. \]

%\clearpage
According to the definition of the derivative, there are:
\begin{align*}
\frac{{\partial \left( {\exp \left( {{ \boldsymbol{\phi} ^ \wedge }} \right) \bm{p}} \right)}}{{\partial \boldsymbol{\phi} }} &= \mathop {\lim }\limits_{\delta \boldsymbol{\phi}  \to \bm{0}} \frac{{\exp \left( {{{\left( {\boldsymbol{\phi}  + \delta \boldsymbol{\phi} } \right)}^ \wedge }} \right) \bm{p} - \exp \left( {{\boldsymbol{\phi} ^ \wedge }} \right)\bm{p}}}{{\delta \boldsymbol{\phi} }}\\
& = \mathop {\lim }\limits_{\delta \boldsymbol{\phi}  \to \bm{0}} \frac{{\exp \left( {{{\left( {{\bm{J}_l}\delta \boldsymbol{\phi} } \right)}^ \wedge }} \right)\exp \left( {{\boldsymbol{\phi} ^ \wedge }} \right) \bm{p} - \exp \left( {{\boldsymbol{\phi} ^ \wedge }} \right) \bm{p}}}{{\delta \boldsymbol{\phi} }}\\
&= \mathop {\lim }\limits_{\delta \boldsymbol{\phi}  \to \bm{0}} \frac{{\left( { \bm{I} + {{\left( {{ \bm{J}_l}\delta \boldsymbol{\phi} } \right)}^ \wedge }} \right)\exp \left( {{\boldsymbol{\phi} ^ \wedge }} \right) \bm{p} - \exp \left( {{\boldsymbol{\phi} ^ \wedge }} \right)\bm{p}}}{{\delta \boldsymbol{\phi} }}\\
&= \mathop {\lim }\limits_{\delta \boldsymbol{\phi}  \to \bm{0}} \frac{{{{\left( {{\bm{J}_l}\delta \boldsymbol{\phi} } \right)}^ \wedge }\exp \left( {{\boldsymbol{\phi} ^ \wedge }} \right)\bm{p}}}{{\delta \boldsymbol{\phi} }}\\
&= \mathop {\lim }\limits_{\delta \boldsymbol{\phi}  \to \bm{0}} \frac{{ - {{\left( {\exp \left( {{\boldsymbol{\phi} ^ \wedge }} \right)\bm{p}} \right)}^ \wedge }{\bm{J}_l}\delta \boldsymbol{\phi} }}{{\delta \boldsymbol{\phi}}} =  - {\left( {\bm{Rp}} \right)^ \wedge }{\bm{J}_l}.
\end{align*}

The approximation of the second row is a linear approximation of BCH, the third behavior is Taylor's approximation after rounding off the high-order term (but because the limit is taken, the equal sign can be written), and the fourth row to the fifth row treat the antisymmetric symbol as a cross product. , after the exchange, change the number. Thus, we derive the derivative of the rotated point relative to the Lie algebra:

\begin{equation}
\frac{{\partial \left( { \bm{Rp}} \right)}}{{\partial \boldsymbol{\phi} }} = {\left( { - \bm{Rp}} \right)^ \wedge }{\bm{J}_l}.
\end{equation}

However, since there is still a more complicated form of $\bm{J}_l$, we don't want to calculate it. The perturbation model described below provides a simpler way to calculate derivatives.


\subsection{Lie algebra$\mathfrak{so}(3)$}

The previously mentioned $\boldsymbol{\phi}$ is actually a kind of Lie algebra.
The Lie algebra corresponding to $\mathrm{SO}(3)$ is a vector defined on $\mathbb{R}^3$, which we remember as $\boldsymbol{\phi}$.
According to the previous derivation, each $\boldsymbol{\phi}$ can generate an antisymmetric matrix:

\begin{equation}
\label{eq:phi}
\boldsymbol{\varPhi} = \boldsymbol{\phi}^{\wedge} = \left[ {\begin{array}{*{20}{c}}
0&{ - {\phi _3}}&{{\phi _2}}\\
{{\phi _3}}&0&{ - {\phi _1}}\\
{ - {\phi _2}}&{{\phi _1}}&0
\end{array}} \right] \in \mathbb{R}^{3 \times 3}.
\end{equation}

Under this definition, the two vectors $\boldsymbol{\phi}_1, \boldsymbol{\phi}_2$'s Li brackets are

\begin{equation}
[\boldsymbol{\phi}_1, \boldsymbol{\phi}_2] = \left( \bm{ \varPhi }_1 \bm{ \varPhi }_2 - \bm{ \varPhi }_2 \bm{ \varPhi }_1 \right)^\vee.
\end{equation}

The reader can verify that the Lie brackets under this definition satisfy the above properties. Since the vector $\boldsymbol{\phi}$ is one-to-one with the antisymmetric matrix, the element of $\mathfrak{so}(3)$ is a three-dimensional vector or a three-dimensional antisymmetric matrix without causing ambiguity. , no difference:
\begin{equation}
\mathfrak{so}(3) = \left\{ \boldsymbol{\phi} \in \mathbb{R}^3, \boldsymbol{\varPhi} = \boldsymbol{\phi^\wedge} \in \mathbb{ R}^{3 \times 3} \right\}.
\end{equation}
Some books also use the symbol $\widehat{\boldsymbol{\phi}}$ to indicate opposition, but the meaning is the same.
At this point, we have made clear the contents of $\mathfrak{so}(3)$.
They are a set of \textbf{3D vectors}, each vector corresponding to an antisymmetric matrix, which can be used to express the derivative of the rotation matrix.

Its relationship to $\mathrm{SO}(3)$ is given by the index map:

\begin{equation}
\bm{R} = \exp ( \boldsymbol{\phi}^\wedge ).
\end{equation}

The index map will be introduced later.
Since we have introduced $\mathfrak{so}(3)$, we will first look at the corresponding Lie algebra on $\mathrm{SE}(3)$.

