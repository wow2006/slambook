\subsection{Disturbance model (left multiplication)}

Another way to do this is to perturb $\bm{R}$ for $\Delta \bm{R}$ and see the rate of change of the result relative to the disturbance. This disturbance can be multiplied on the left or multiplied on the right. The final result will be slightly different. Let's take the left disturbance as an example. Let the left perturbation $\Delta \bm{R}$ correspond to the Lie algebra as $\boldsymbol{\varphi}$. Then, for $\boldsymbol{\varphi}$, that is:

\begin{equation}
\frac{{\partial \left( {\bm{Rp}} \right)}}{{\partial \boldsymbol{\varphi} }} = \mathop {\lim }\limits_{\boldsymbol{\varphi}  \to \bm{0}} \frac{{\exp \left( {{\boldsymbol{\varphi} ^ \wedge }} \right)\exp \left( {{\boldsymbol{\phi} ^ \wedge }} \right)\bm{p} - \exp \left( {{\boldsymbol{\phi} ^ \wedge }} \right)\bm{p}}}{\boldsymbol{\varphi} }.
\end{equation}

The derivation of this formula is simpler than the above:

\begin{align*}
\frac{{\partial \left( {\bm{Rp}} \right)}}{{\partial \boldsymbol{\varphi} }} &= \mathop {\lim }\limits_{\boldsymbol{\varphi}  \to \bm{0}} \frac{{\exp \left( {{\boldsymbol{\varphi} ^ \wedge }} \right)\exp \left( {{\boldsymbol{\phi} ^ \wedge }} \right)\bm{p} - \exp \left( {{\boldsymbol{\phi} ^ \wedge }} \right)\bm{p}}}{ \boldsymbol{\varphi} }\\
&= \mathop {\lim }\limits_{\boldsymbol{\varphi } \to \bm{0}} \frac{{\left( {\bm{I} + {\boldsymbol{\varphi }^ \wedge }} \right)\exp \left( {{\boldsymbol{\phi} ^ \wedge }} \right)\bm{p} - \exp \left( {{\boldsymbol{\phi} ^ \wedge }} \right)\bm{p}}}{\boldsymbol{\varphi} }\\
&= \mathop {\lim }\limits_{\boldsymbol{\varphi}  \to \bm{0}} \frac{{{\boldsymbol{\varphi} ^ \wedge }\bm{Rp}}}{\boldsymbol{\varphi} } = \mathop {\lim }\limits_{\boldsymbol{\varphi}  \to \bm{0}} \frac{{ - {{\left( \bm{Rp} \right)}^ \wedge }\boldsymbol{\varphi} }}{\boldsymbol{\varphi} } =  - {\left( \bm{Rp} \right)^ \wedge }.
\end{align*}

It can be seen that the calculation of a Jacobian $\bm{J}_l$ is omitted compared to the direct derivation of Lie algebra. This makes the perturbation model more practical. The reader must understand the derivative operation here, which is of great significance in pose estimation.