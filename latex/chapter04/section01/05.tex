\subsection{Lie algebra $\mathfrak{se}(3)$}

For $\mathrm{SE}(3)$, it also has a corresponding Lie algebra $\mathfrak{se}(3)$. To save space, I won't cover how to bring $\mathfrak{se}(3)$. Similar to $\mathfrak{so}(3)$, $\mathfrak{se}(3)$ is located in the $\mathbb{R}^6$ space:

\begin{equation}
\mathfrak{se}(3) = \left\{ { \boldsymbol{\xi} = \left[ \begin{array}{l}
	\boldsymbol{\rho} \\
	\boldsymbol{\phi} 
	\end{array} \right]
	 \in { \mathbb{R}^6} ,
	 \boldsymbol{\rho} \in { \mathbb{R}^3}, \boldsymbol{\phi} \in \mathfrak{so} \left( 3 \right),{ \boldsymbol{\xi} ^ \wedge } = \left[ {\begin{array}{*{20}{c}}
		{{ \boldsymbol{\phi} ^ \wedge }}& \boldsymbol{\rho} \\
		{{\bm{0}^\mathrm{T}}}&0
		\end{array}} \right] \in { \mathbb{R}^{4 \times 4}}} \right\}.
\end{equation}

We write each $\mathfrak{se}(3)$ element as $\boldsymbol{\xi}$, which is a six-dimensional vector. The first three dimensions are translation (but the meaning is different from the translation in the transformation matrix, the analysis is seen later), which is denoted as $\boldsymbol{\rho}$; after the three-dimensional rotation, it is recorded as $\boldsymbol{\phi}$, which is essentially $\mathfrak{so}(3)$Element\footnote{Please note that some places put the rotation in front and the translation in the back, it is also possible. In the program, it doesn't matter, they are stored in a structure. }. At the same time, we extended the meaning of the $^\wedge$ symbol. In $\mathfrak{se}(3)$, a six-dimensional vector is also converted to a four-dimensional matrix using the $^\wedge$ symbol, but the anti-symbol is no longer represented here:

\begin{equation}
{ \boldsymbol{\xi} ^ \wedge } = \left[ {\begin{array}{*{20}{c}}
	{{ \boldsymbol{\phi} ^ \wedge }}& \boldsymbol{\rho} \\
	{{\bm{0}^\mathrm{T}}}&0
	\end{array}} \right] \in { \mathbb{R}^{4 \times 4}}.
\end{equation}

We still use the $^\wedge$ and $^\vee$ symbols to refer to the relationship from "vector to matrix" and "matrix to vector" to maintain consistency with $\mathfrak{so}(3)$ Sex. They are still one-to-one correspondence. The reader can simply interpret $\mathfrak{se}(3)$ as a "vector consisting of a translation plus a $\mathfrak{so}(3)$ element" (although $\boldsymbol{\rho} here) $ is not directly translated). Similarly, the Lie algebra $\mathfrak{se}(3)$ also has a Lee bracket similar to $\mathfrak{so}(3)$:

\begin{equation}
[ \boldsymbol{\xi}_1, \boldsymbol{\xi}_2 ] = \left( \boldsymbol{\xi}_1^\wedge \boldsymbol{\xi}_2^\wedge -\boldsymbol{\xi}_2^ \wedge \boldsymbol{\xi}_1^\wedge \right) ^\vee.
\end{equation}

The reader can verify that it satisfies the definition of Lie algebra (reserved as an exercise). So far we have seen two important Lie algebras $\mathfrak{so}(3)$ and $\mathfrak{se}(3)$.
