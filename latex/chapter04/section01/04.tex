\subsection{Lie algebra$\mathfrak{so}(3)$}

The previously mentioned $\boldsymbol{\phi}$ is actually a kind of Lie algebra.
The Lie algebra corresponding to $\mathrm{SO}(3)$ is a vector defined on $\mathbb{R}^3$, which we remember as $\boldsymbol{\phi}$.
According to the previous derivation, each $\boldsymbol{\phi}$ can generate an antisymmetric matrix:

\begin{equation}
\label{eq:phi}
\boldsymbol{\varPhi} = \boldsymbol{\phi}^{\wedge} = \left[ {\begin{array}{*{20}{c}}
0&{ - {\phi _3}}&{{\phi _2}}\\
{{\phi _3}}&0&{ - {\phi _1}}\\
{ - {\phi _2}}&{{\phi _1}}&0
\end{array}} \right] \in \mathbb{R}^{3 \times 3}.
\end{equation}

Under this definition, the two vectors $\boldsymbol{\phi}_1, \boldsymbol{\phi}_2$'s Li brackets are

\begin{equation}
[\boldsymbol{\phi}_1, \boldsymbol{\phi}_2] = \left( \bm{ \varPhi }_1 \bm{ \varPhi }_2 - \bm{ \varPhi }_2 \bm{ \varPhi }_1 \right)^\vee.
\end{equation}

The reader can verify that the Lie brackets under this definition satisfy the above properties. Since the vector $\boldsymbol{\phi}$ is one-to-one with the antisymmetric matrix, the element of $\mathfrak{so}(3)$ is a three-dimensional vector or a three-dimensional antisymmetric matrix without causing ambiguity. , no difference:
\begin{equation}
\mathfrak{so}(3) = \left\{ \boldsymbol{\phi} \in \mathbb{R}^3, \boldsymbol{\varPhi} = \boldsymbol{\phi^\wedge} \in \mathbb{ R}^{3 \times 3} \right\}.
\end{equation}
Some books also use the symbol $\widehat{\boldsymbol{\phi}}$ to indicate opposition, but the meaning is the same.
At this point, we have made clear the contents of $\mathfrak{so}(3)$.
They are a set of \textbf{3D vectors}, each vector corresponding to an antisymmetric matrix, which can be used to express the derivative of the rotation matrix.

Its relationship to $\mathrm{SO}(3)$ is given by the index map:

\begin{equation}
\bm{R} = \exp ( \boldsymbol{\phi}^\wedge ).
\end{equation}

The index map will be introduced later.
Since we have introduced $\mathfrak{so}(3)$, we will first look at the corresponding Lie algebra on $\mathrm{SE}(3)$.

