\subsection{Data demonstration of the Eigen geometry module}

Now, let's actually practice the various rotation expressions mentioned earlier. We will use quaternions, Euler angles, and rotation matrices in Eigen to demonstrate how they are transformed. We will also give a visualization program to help the reader understand the relationship of these transformations.

\begin{lstlisting}[language=c++,caption=slambook2/ch3/useGeometry/useGeometry.cpp]
#include <iostream>
#include <cmath>
using namespace std;

#include <Eigen/Core>
#include <Eigen/Geometry>

using namespace Eigen;
// This program demonstrates how to use the Eigen geometry module

int main(int argc, char **argv) {
    // The Eigen/Geometry module provides a variety of rotation and translation representations
    // 3D rotation matrix directly using Matrix3d or Matrix3f
    Matrix3d rotation_matrix = Matrix3d::Identity();
    // The rotation vector uses AngleAxis, the underlying layer is not directly Matrix, but the operation can be treated as a matrix (because the operator is overloaded)
    AngleAxisd rotation_vector(M_PI / 4, Vector3d(0, 0, 1)); //Rotate 45 degrees along the Z axis
    cout.precision(3);
    cout << "rotation matrix = \n " << rotation_vector.matrix() << endl; //convert to matrix with matrix()
    // can also be assigned directly
    rotation_matrix = rotation_vector.toRotationMatrix();
    // coordinate transformation with AngleAxis
    Vector3d v(1, 0, 0);
    Vector3d v_rotated = rotation_vector * v;
    cout << "(1,0,0) after rotation (by angle axis) = " << v_rotated.transpose() << endl;
    // Or use a rotation matrix
    v_rotated = rotation_matrix * v;
    cout << "(1,0,0) after rotation (by matrix) = " << v_rotated.transpose() << endl;
    
    // Euler angle: You can convert the rotation matrix directly into Euler angles
    Vector3d euler_angles = rotation_matrix.eulerAngles(2, 1, 0); // ZYX order, ie roll pitch yaw order
    cout << "yaw pitch roll = " << euler_angles.transpose() << endl;
    
    // Euclidean transformation matrix using Eigen::Isometry
    Isometry3d T = Isometry3d::Identity(); // Although called 3d, it is essentially a 4*4 matrix
    T.rotate(rotation_vector); // Rotate according to rotation_vector
    T.pretranslate(Vector3d(1, 3, 4)); // Set the translation vector to (1,3,4)
    cout << "Transform matrix = \n" << T.matrix() << endl;
    
    / / Use the transformation matrix for coordinate transformation
    Vector3d v_transformed = T * v; // Equivalent to R*v+t
    cout << "v tranformed = " << v_transformed.transpose() << endl;
    
    // For affine and projective transformations, use Eigen::Affine3d and Eigen::Projective3d.
    
    // Quaternion
    // You can assign AngleAxis directly to quaternions, and vice versa
    Quaterniond q = Quaterniond(rotation_vector);
    cout << "quaternion from rotation vector = " << q.coeffs().transpose()
    << endl; // Note that the order of coeffs is (x, y, z, w), w is the real part, the first three are the imaginary part
    // can also assign a rotation matrix to it
    q = Quaterniond(rotation_matrix);
    cout << "quaternion from rotation matrix = " << q.coeffs().transpose() << endl;
    // Rotate a vector with a quaternion and use overloaded multiplication
    V_rotated = q * v; // Note that the math is qvq^{-1}
    cout << "(1,0,0) after rotation = " << v_rotated.transpose() << endl;
    // expressed by regular vector multiplication, it should be calculated as follows
    cout << "should be equal to " << (q * Quaterniond(0, 1, 0, 0) * q.inverse()).coeffs().transpose() << endl;
    
    return 0;
}
\end{lstlisting}

The various forms of expression in Eigen are summarized below. Note that each type has both single and double data types and, as before, cannot be automatically converted by the compiler. Taking double precision as an example, you can change the last d to f, which is a single-precision data structure.
\begin{itemize}
	\item rotation matrix ( $ 3  \times  3 $ ): Eigen::Matrix3d.
	\item rotation vector ( $ 3  \times  1 $ ): Eigen::AngleAxisd.
	\item Euler angle ( $ 3  \times  1 $ ): Eigen::Vector3d.
	\item quaternion ( $ 4  \times  1 $ ): Eigen::Quaterniond.
	\item Euclidean transformation matrix ( $ 4  \times  4 $ ): Eigen::Isometry3d.
	\item affine transform ( $ 4  \times  4 $ ): Eigen::Affine3d.
	\item projective transformation ( $ 4  \times  4 $ ): Eigen::Projective3d.
\end{itemize}

This program can be compiled by referring to the corresponding CMakeLists in the code. In this program, I demonstrate how to use the rotation matrix, rotation vectors (AngleAxis), Euler angles, and quaternions in Eigen. We use these rotations to rotate a vector $ \bm {v} $ and find that the result is the same (different is really a hell of it). At the same time, it also demonstrates how to convert these expressions in the program. Readers who want to learn more about Eigen's geometry modules can refer to it ( \url {http://eigen.tuxfamily.org/dox/group__TutorialGeometry.html}).

The reader is cautioned that the \textbf {program code has some subtle differences from the mathematical representation}. For example, by operator overloading, quaternions and three-dimensional vectors can directly calculate multiplication, but mathematically, the vector needs to be converted into a virtual quaternion, and then quaternion multiplication is used for calculation. The same applies to the transformation matrix. Multiply the case of a three-dimensional vector. In general, the usage in the program is more flexible than the mathematical formula.