\chapter{Cameras and Images}
\label{cpt:5}

\begin{mdframed}  
	\textbf{main target}
	\begin{enumerate}
		\item Understand the model of the pinhole camera and participate in the radial distortion parameters.
		\item Understand how a spatial point is projected onto the camera's imaging plane.
		\item Master the image storage and expression of OpenCV.
		\item Learn the basic camera calibration method.
	\end{enumerate}
\end{mdframed}

In the previous two lectures, we introduced the question of how the robot expresses its own pose, and partially explained the meaning of the variables and the equations of motion in the classical model of SLAM. This talk will discuss "How robots observe the outside world," which is part of the observation equation. In the camera-based visual SLAM, the observation mainly refers to the process of \textbf{camera imaging}.

We can see a lot of photos in real life. In a computer, a photo consists of a number of pixels, each of which records information about color or brightness. Light reflected or emitted by an object in the three-dimensional world passes through the camera's optical center and is projected onto the imaging plane of the camera. After the camera's light sensor receives the light, it produces a measurement and gets the pixels, which form the photo we see. Can this process be described by mathematical principles? This lecture will first discuss the camera model, explain how the projection relationship is specifically described, and what is the internal reference of the camera. At the same time, a brief introduction to the principle of binocular imaging and RGB-D cameras. Then, introduce the basic operation of two-dimensional photo pixels. Finally, an experiment of point cloud stitching is demonstrated based on the meaning of internal and external parameters.
\newpage

\includepdf{chapter05/resources/other/ch5.pdf}
\newpage

\section{Practice: Eigen Geometry Module}

\subsection{BCH formula and approximation}

A major motivation for using Lie algebra is to optimize, and the derivative is a very necessary information in the optimization process (we will cover it in detail in Lecture 6). Let's consider a problem below. Although we have already understood the relationship between Lie group and Lie algebra on $\mathrm{SO}(3)$ and $\mathrm{SE}(3)$, but in $\mathrm{SO}(3)$ What happens to $\mathfrak{so}(3)$ in Lie algebra when two matrix multiplications are completed? Conversely, when $\mathfrak{so}(3)$ is used to add two Lie algebras, does $\mathrm{SO}(3)$ correspond to the product of the two matrices? If established, it is equivalent to:

\[
  \exp \left( {\boldsymbol{\phi} _1^ \wedge } \right)\exp \left( {\boldsymbol{\phi} _2^ \wedge } \right) = \exp \left( {{{\left( {{\boldsymbol{\phi} _1} + {\boldsymbol{\phi} _2}} \right)}^ \wedge }} \right) ?
\]

If $\boldsymbol{\phi}_1, \boldsymbol{\phi}_2$ is a scalar, then obviously this is true; but here we calculate the exponential function of \textbf{matrix} instead of the scalar exponent. In other words, we are studying whether the following formula holds:

\[
  \ln \left( \exp \left( \bm{A} \right) \exp \left( \bm{B} \right) \right) = \bm{A} + \bm{B} \; ?
\]

Unfortunately, this formula does not hold true in the matrix. The complete form of the product of the two Lie algebra indices is represented by the Baker-Campbell-Hausdorff formula (BCH formula)\footnote{ see \url{https://en.wikipedia.org/wiki/Baker-Campbell-Hausdorff\_formula}. } given. Due to the complexity of its complete form, we only give the first few items of its expansion:

\begin{equation}
  \ln \left( {\exp \left( \bm{A} \right)\exp \left( \bm{B} \right)} \right) = \bm{A} + \bm{B} + \frac{1}{2}\left[ {\bm{A}, \bm{B}} \right] + \frac{1}{{12}}\left[ {\bm{A},\left[ {\bm{A}, \bm{B}} \right]} \right] - \frac{1}{{12}}\left[ {\bm{B},\left[ {\bm{A},\bm{B}} \right]} \right] +  \cdots
\end{equation}

Where $[]$ is the brackets. The BCH formula tells us that when dealing with the product of two matrix indices, they produce some remainders consisting of Lie brackets. In particular, consider the Lie algebra on $\mathrm{SO}(3)$$\ln { \left( {\exp \left( { \boldsymbol{\phi} _1^ \wedge } \right)\exp \left ( {\boldsymbol{\phi} _2^ \wedge } \right)} \right) ^ \vee }$, when $\boldsymbol{\phi_1}$ or $\boldsymbol{\phi_2}$ is small, small Items with more than two quantities can be ignored. At this time, BCH has a linear approximation \footnote{ concrete derivation of BCH specific form and approximate expression, this book is not discussed, please refer to the document \cite{Barfoot2016}. }:

\begin{equation}
  \ln { \left( {\exp \left( { \boldsymbol{\phi} _1^ \wedge } \right)\exp \left( {\boldsymbol{\phi} _2^ \wedge } \right)} \right) ^ \vee } \approx \left\{
    \begin{array}{l}
      {\bm{J}_l}{\left( {{\boldsymbol{\phi} _2}} \right)^{ - 1}}{ \boldsymbol{\phi} _1} + {\boldsymbol{\phi} _2} \quad \text{when} \boldsymbol{\phi}_1 \text{Small amount},\\
      {\bm{J}_r}{\left( {{\boldsymbol{\phi} _1}} \right)^{ - 1}}{\boldsymbol{\phi} _2} + {\boldsymbol{\phi} _1} \quad \text{when} \boldsymbol{\phi}_2 \text{Small amount}.
  \end{array} \right.
\end{equation}

Take the first approximation as an example. This formula tells us to multiply a tiny rotation matrix $\bm{R}_1$ to a rotation matrix $\bm{R}_2$ (the Lie algebra is $\boldsymbol{\phi}_2$) (Li algebra is $\boldsymbol{\phi} _1$) can be approximated as the original Lie algebra $\boldsymbol{\phi}_2$Added an item${\bm{J}_l}{\left( {{\boldsymbol{\phi} _2}} \right)^{ - 1}}{ \boldsymbol{\phi} _1}$. 
Similarly, the second approximation describes the case where the right multiplied by a small displacement. Therefore, under the BCH approximation, the Lie algebra is divided into a left-multiplying approximation and a right-multiplying approximation. In use, we must pay attention to whether the left-multiplier model or the right-multiply model is used.

This book takes the left multiplication as an example. Left multiply BCH approximates Jacobian $\bm{J}_l$ is actually the content of the form ~\eqref{eq:lieAlgebraJacobian}~:

\begin{equation} 
{ \bm{J}_l} = \bm{J} = \frac{{\sin \theta }}{\theta } \bm{I} + \left( {1 - \frac{{\sin \theta }}{\theta }} \right) \bm{a} { \bm{a}^\mathrm{T}} + \frac{{1 - \cos \theta }}{\theta }{ \bm{a}^ \wedge}.
\end{equation}

Its inverse is:

\begin{equation}
\bm{J}_l^{ - 1} = \frac{\theta }{2}\cot \frac{\theta }{2} \bm{I} + \left( {1 - \frac{\theta } {2}\cot \frac{\theta }{2}} \right) \bm{a} {\bm{a}^\mathrm{T}} - \frac{\theta }{2}{ \bm{ a}^ \wedge }.
\end{equation}

Right-handed Jacobi only needs to take a negative sign for the argument:

\begin{equation}
\bm{J}_r(\boldsymbol{\phi}) =\bm{J}_l(-\boldsymbol{\phi}) .
\end{equation}

In this way, we can talk about the relationship between Lie group multiplication and Lie algebra addition.

For the convenience of the reader, we restate the meaning of the BCH approximation. Suppose that for a rotation of $\bm{R}$, the corresponding Lie algebra is $\boldsymbol{\phi}$. We give it a small rotation to the left, denoted as $\Delta \bm{R}$, and the corresponding Lie algebra is $\Delta \boldsymbol{\phi}$. Then, on Lie group, the result is $ \Delta \bm{R} \cdot \bm{R}$, and on the Lie algebra, according to the BCH approximation, it is $\bm{J}_l^{-1 } (\boldsymbol{\phi}) \Delta \boldsymbol{\phi} + \boldsymbol{\phi}$. Combined, you can simply write:

\begin{equation}
\exp \left( {\Delta { \boldsymbol{\phi} ^ \wedge }} \right)\exp \left( {{ \boldsymbol{\phi} ^ \wedge }} \right) = \exp \left( {{{\left( { \boldsymbol{\phi} + \bm{J}_l^{ - 1}\left( \boldsymbol{\phi} \right)\Delta \boldsymbol{\phi} } \right)} ^ \wedge }} \right).
\end{equation}

Conversely, if we add on Lie algebra and add $\boldsymbol{\phi}$ to $\Delta \boldsymbol{\phi}$, we can approximate the multiplication of the left and right Jacobian on the Lie group:

\begin{equation}
\exp \left( {{{\left( { \boldsymbol{\phi}  + \Delta \boldsymbol{\phi} } \right)}^ \wedge }} \right) = \exp \left( {{{\left( {{ \bm{J}_l}\Delta \boldsymbol{\phi} } \right)}^ \wedge }} \right)\exp \left( {{ \boldsymbol{\phi} ^ \wedge }} \right) = \exp \left( {{\boldsymbol{\phi} ^ \wedge }} \right)\exp \left( {{{\left( {{\bm{J}_r}\Delta \boldsymbol{\phi} } \right)}^ \wedge }} \right).
\end{equation}

This provides a theoretical basis for calculus after Lie algebra. Similarly, for $\mathrm{SE}(3)$, there is a similar BCH approximation:

\begin{equation}
\exp \left( {\Delta {\boldsymbol{\xi} ^ \wedge }} \right)\exp \left( {{ \boldsymbol{\xi} ^ \wedge }} \right) \approx \exp \left ( {{{\left( {{ \bm{\mathcal{J}}_l^{-1} }\Delta \boldsymbol{\xi} + \boldsymbol{\xi} } \right)}^ \wedge }} \right),
\end{equation}

\begin{equation}
\exp \left( {{ \boldsymbol{\xi} ^ \wedge }} \right) \exp \left( {\Delta {\boldsymbol{\xi} ^ \wedge }} \right) \approx \exp \left ( {{{\left( {{ \bm{\mathcal{J}}_r^{-1} }\Delta \boldsymbol{\xi} + \boldsymbol{\xi} } \right)}^ \wedge }} \right).
\end{equation}

Here the $\bm{\mathcal{J}}_l$ form is more complicated. It is a matrix of $6 \times 6$. Readers can refer to the contents of the \cite{Barfoot2016} formulas (7.82) and (7.83). Since we did not use the Jacobi in the calculation, the actual form is omitted here.


\subsubsection{Install OpenCV}

OpenCV\footnote{Official homepage:\url{http://opencv.org}. } provides a large number of open source image algorithms, is a library of image processing algorithms used in computer vision. This book also uses OpenCV for basic image processing. Before using it, it is recommended that the reader install it from the source code. Under Ubuntu, there are two ways to \textbf{install from source code} and \textbf{install library file only}:

\begin{enumerate}
	\item Installation from source code means downloading all OpenCV source code from the OpenCV website and compiling and installing it on the machine for use. The advantage is that the version you can choose is rich, and you can see the source code, but it takes some compilation time.
	\item only installs the library file, which means to install the library file compiled by Ubuntu community personnel through Ubuntu, so there is no need to recompile it.
\end{enumerate}

Since we are using a newer version of OpenCV, we must install it from the source code. As a result, you can adjust some compile options to match the programming environment (for example, GPU acceleration is not required); in addition, source code installation can use some additional features. OpenCV currently maintains two major versions, divided into OpenCV 2.4 series and OpenCV 3 series. This book uses the OpenCV \textbf{3} series.

Since the OpenCV project is relatively large, it is not placed under the 3rdparty of this book. Readers can download from ~\url{http://opencv.org/downloads.html} and choose OpenCV for Linux. You will get a tarball like opencv-3.1.0.zip. Extract it to any directory, we found that OpenCV is also a cmake project.

Before compiling, first install the OpenCV dependencies:

\begin{lstlisting}[language=sh,caption=terminal input:]
Sudo apt-get install build-essential libgtk2.0-dev libvtk5-dev libjpeg-dev libtiff4-dev libjasper-dev libopenexr-dev libtbb-dev
\end{lstlisting}

In fact, OpenCV has a lot of dependencies, and the lack of some compiled items will affect some of its features (but we won't use all the features). OpenCV will check if the dependencies will be installed and adjust its functionality during the cmake phase. If you have a GPU on your computer and have related dependencies installed, OpenCV will also speed up the GPU. But for this book, the above dependencies are enough.

Subsequent compilation and installation is the same as the normal cmake project. After make, call sudo make install to install OpenCV on your machine (instead of just compiling it). Depending on the machine configuration, this compilation process can take anywhere from twenty minutes to an hour. If your CPU is strong, you can use a command like "make -j4" to call multiple threads to compile (the arguments after -j are the number of threads used). After installation, OpenCV is stored by default in the /usr/local directory. You can find the location where OpenCV header files and library files are installed and see where they are. In addition, if you have previously installed the OpenCV 2 series, it is recommended that you install OpenCV 3 somewhere else (think how this should work).



\section{Practice: Eigen Geometry Module}

\subsection{BCH formula and approximation}

A major motivation for using Lie algebra is to optimize, and the derivative is a very necessary information in the optimization process (we will cover it in detail in Lecture 6). Let's consider a problem below. Although we have already understood the relationship between Lie group and Lie algebra on $\mathrm{SO}(3)$ and $\mathrm{SE}(3)$, but in $\mathrm{SO}(3)$ What happens to $\mathfrak{so}(3)$ in Lie algebra when two matrix multiplications are completed? Conversely, when $\mathfrak{so}(3)$ is used to add two Lie algebras, does $\mathrm{SO}(3)$ correspond to the product of the two matrices? If established, it is equivalent to:

\[
  \exp \left( {\boldsymbol{\phi} _1^ \wedge } \right)\exp \left( {\boldsymbol{\phi} _2^ \wedge } \right) = \exp \left( {{{\left( {{\boldsymbol{\phi} _1} + {\boldsymbol{\phi} _2}} \right)}^ \wedge }} \right) ?
\]

If $\boldsymbol{\phi}_1, \boldsymbol{\phi}_2$ is a scalar, then obviously this is true; but here we calculate the exponential function of \textbf{matrix} instead of the scalar exponent. In other words, we are studying whether the following formula holds:

\[
  \ln \left( \exp \left( \bm{A} \right) \exp \left( \bm{B} \right) \right) = \bm{A} + \bm{B} \; ?
\]

Unfortunately, this formula does not hold true in the matrix. The complete form of the product of the two Lie algebra indices is represented by the Baker-Campbell-Hausdorff formula (BCH formula)\footnote{ see \url{https://en.wikipedia.org/wiki/Baker-Campbell-Hausdorff\_formula}. } given. Due to the complexity of its complete form, we only give the first few items of its expansion:

\begin{equation}
  \ln \left( {\exp \left( \bm{A} \right)\exp \left( \bm{B} \right)} \right) = \bm{A} + \bm{B} + \frac{1}{2}\left[ {\bm{A}, \bm{B}} \right] + \frac{1}{{12}}\left[ {\bm{A},\left[ {\bm{A}, \bm{B}} \right]} \right] - \frac{1}{{12}}\left[ {\bm{B},\left[ {\bm{A},\bm{B}} \right]} \right] +  \cdots
\end{equation}

Where $[]$ is the brackets. The BCH formula tells us that when dealing with the product of two matrix indices, they produce some remainders consisting of Lie brackets. In particular, consider the Lie algebra on $\mathrm{SO}(3)$$\ln { \left( {\exp \left( { \boldsymbol{\phi} _1^ \wedge } \right)\exp \left ( {\boldsymbol{\phi} _2^ \wedge } \right)} \right) ^ \vee }$, when $\boldsymbol{\phi_1}$ or $\boldsymbol{\phi_2}$ is small, small Items with more than two quantities can be ignored. At this time, BCH has a linear approximation \footnote{ concrete derivation of BCH specific form and approximate expression, this book is not discussed, please refer to the document \cite{Barfoot2016}. }:

\begin{equation}
  \ln { \left( {\exp \left( { \boldsymbol{\phi} _1^ \wedge } \right)\exp \left( {\boldsymbol{\phi} _2^ \wedge } \right)} \right) ^ \vee } \approx \left\{
    \begin{array}{l}
      {\bm{J}_l}{\left( {{\boldsymbol{\phi} _2}} \right)^{ - 1}}{ \boldsymbol{\phi} _1} + {\boldsymbol{\phi} _2} \quad \text{when} \boldsymbol{\phi}_1 \text{Small amount},\\
      {\bm{J}_r}{\left( {{\boldsymbol{\phi} _1}} \right)^{ - 1}}{\boldsymbol{\phi} _2} + {\boldsymbol{\phi} _1} \quad \text{when} \boldsymbol{\phi}_2 \text{Small amount}.
  \end{array} \right.
\end{equation}

Take the first approximation as an example. This formula tells us to multiply a tiny rotation matrix $\bm{R}_1$ to a rotation matrix $\bm{R}_2$ (the Lie algebra is $\boldsymbol{\phi}_2$) (Li algebra is $\boldsymbol{\phi} _1$) can be approximated as the original Lie algebra $\boldsymbol{\phi}_2$Added an item${\bm{J}_l}{\left( {{\boldsymbol{\phi} _2}} \right)^{ - 1}}{ \boldsymbol{\phi} _1}$. 
Similarly, the second approximation describes the case where the right multiplied by a small displacement. Therefore, under the BCH approximation, the Lie algebra is divided into a left-multiplying approximation and a right-multiplying approximation. In use, we must pay attention to whether the left-multiplier model or the right-multiply model is used.

This book takes the left multiplication as an example. Left multiply BCH approximates Jacobian $\bm{J}_l$ is actually the content of the form ~\eqref{eq:lieAlgebraJacobian}~:

\begin{equation} 
{ \bm{J}_l} = \bm{J} = \frac{{\sin \theta }}{\theta } \bm{I} + \left( {1 - \frac{{\sin \theta }}{\theta }} \right) \bm{a} { \bm{a}^\mathrm{T}} + \frac{{1 - \cos \theta }}{\theta }{ \bm{a}^ \wedge}.
\end{equation}

Its inverse is:

\begin{equation}
\bm{J}_l^{ - 1} = \frac{\theta }{2}\cot \frac{\theta }{2} \bm{I} + \left( {1 - \frac{\theta } {2}\cot \frac{\theta }{2}} \right) \bm{a} {\bm{a}^\mathrm{T}} - \frac{\theta }{2}{ \bm{ a}^ \wedge }.
\end{equation}

Right-handed Jacobi only needs to take a negative sign for the argument:

\begin{equation}
\bm{J}_r(\boldsymbol{\phi}) =\bm{J}_l(-\boldsymbol{\phi}) .
\end{equation}

In this way, we can talk about the relationship between Lie group multiplication and Lie algebra addition.

For the convenience of the reader, we restate the meaning of the BCH approximation. Suppose that for a rotation of $\bm{R}$, the corresponding Lie algebra is $\boldsymbol{\phi}$. We give it a small rotation to the left, denoted as $\Delta \bm{R}$, and the corresponding Lie algebra is $\Delta \boldsymbol{\phi}$. Then, on Lie group, the result is $ \Delta \bm{R} \cdot \bm{R}$, and on the Lie algebra, according to the BCH approximation, it is $\bm{J}_l^{-1 } (\boldsymbol{\phi}) \Delta \boldsymbol{\phi} + \boldsymbol{\phi}$. Combined, you can simply write:

\begin{equation}
\exp \left( {\Delta { \boldsymbol{\phi} ^ \wedge }} \right)\exp \left( {{ \boldsymbol{\phi} ^ \wedge }} \right) = \exp \left( {{{\left( { \boldsymbol{\phi} + \bm{J}_l^{ - 1}\left( \boldsymbol{\phi} \right)\Delta \boldsymbol{\phi} } \right)} ^ \wedge }} \right).
\end{equation}

Conversely, if we add on Lie algebra and add $\boldsymbol{\phi}$ to $\Delta \boldsymbol{\phi}$, we can approximate the multiplication of the left and right Jacobian on the Lie group:

\begin{equation}
\exp \left( {{{\left( { \boldsymbol{\phi}  + \Delta \boldsymbol{\phi} } \right)}^ \wedge }} \right) = \exp \left( {{{\left( {{ \bm{J}_l}\Delta \boldsymbol{\phi} } \right)}^ \wedge }} \right)\exp \left( {{ \boldsymbol{\phi} ^ \wedge }} \right) = \exp \left( {{\boldsymbol{\phi} ^ \wedge }} \right)\exp \left( {{{\left( {{\bm{J}_r}\Delta \boldsymbol{\phi} } \right)}^ \wedge }} \right).
\end{equation}

This provides a theoretical basis for calculus after Lie algebra. Similarly, for $\mathrm{SE}(3)$, there is a similar BCH approximation:

\begin{equation}
\exp \left( {\Delta {\boldsymbol{\xi} ^ \wedge }} \right)\exp \left( {{ \boldsymbol{\xi} ^ \wedge }} \right) \approx \exp \left ( {{{\left( {{ \bm{\mathcal{J}}_l^{-1} }\Delta \boldsymbol{\xi} + \boldsymbol{\xi} } \right)}^ \wedge }} \right),
\end{equation}

\begin{equation}
\exp \left( {{ \boldsymbol{\xi} ^ \wedge }} \right) \exp \left( {\Delta {\boldsymbol{\xi} ^ \wedge }} \right) \approx \exp \left ( {{{\left( {{ \bm{\mathcal{J}}_r^{-1} }\Delta \boldsymbol{\xi} + \boldsymbol{\xi} } \right)}^ \wedge }} \right).
\end{equation}

Here the $\bm{\mathcal{J}}_l$ form is more complicated. It is a matrix of $6 \times 6$. Readers can refer to the contents of the \cite{Barfoot2016} formulas (7.82) and (7.83). Since we did not use the Jacobi in the calculation, the actual form is omitted here.


\subsubsection{Install OpenCV}

OpenCV\footnote{Official homepage:\url{http://opencv.org}. } provides a large number of open source image algorithms, is a library of image processing algorithms used in computer vision. This book also uses OpenCV for basic image processing. Before using it, it is recommended that the reader install it from the source code. Under Ubuntu, there are two ways to \textbf{install from source code} and \textbf{install library file only}:

\begin{enumerate}
	\item Installation from source code means downloading all OpenCV source code from the OpenCV website and compiling and installing it on the machine for use. The advantage is that the version you can choose is rich, and you can see the source code, but it takes some compilation time.
	\item only installs the library file, which means to install the library file compiled by Ubuntu community personnel through Ubuntu, so there is no need to recompile it.
\end{enumerate}

Since we are using a newer version of OpenCV, we must install it from the source code. As a result, you can adjust some compile options to match the programming environment (for example, GPU acceleration is not required); in addition, source code installation can use some additional features. OpenCV currently maintains two major versions, divided into OpenCV 2.4 series and OpenCV 3 series. This book uses the OpenCV \textbf{3} series.

Since the OpenCV project is relatively large, it is not placed under the 3rdparty of this book. Readers can download from ~\url{http://opencv.org/downloads.html} and choose OpenCV for Linux. You will get a tarball like opencv-3.1.0.zip. Extract it to any directory, we found that OpenCV is also a cmake project.

Before compiling, first install the OpenCV dependencies:

\begin{lstlisting}[language=sh,caption=terminal input:]
Sudo apt-get install build-essential libgtk2.0-dev libvtk5-dev libjpeg-dev libtiff4-dev libjasper-dev libopenexr-dev libtbb-dev
\end{lstlisting}

In fact, OpenCV has a lot of dependencies, and the lack of some compiled items will affect some of its features (but we won't use all the features). OpenCV will check if the dependencies will be installed and adjust its functionality during the cmake phase. If you have a GPU on your computer and have related dependencies installed, OpenCV will also speed up the GPU. But for this book, the above dependencies are enough.

Subsequent compilation and installation is the same as the normal cmake project. After make, call sudo make install to install OpenCV on your machine (instead of just compiling it). Depending on the machine configuration, this compilation process can take anywhere from twenty minutes to an hour. If your CPU is strong, you can use a command like "make -j4" to call multiple threads to compile (the arguments after -j are the number of threads used). After installation, OpenCV is stored by default in the /usr/local directory. You can find the location where OpenCV header files and library files are installed and see where they are. In addition, if you have previously installed the OpenCV 2 series, it is recommended that you install OpenCV 3 somewhere else (think how this should work).



\section{Practice: Eigen Geometry Module}

\subsection{BCH formula and approximation}

A major motivation for using Lie algebra is to optimize, and the derivative is a very necessary information in the optimization process (we will cover it in detail in Lecture 6). Let's consider a problem below. Although we have already understood the relationship between Lie group and Lie algebra on $\mathrm{SO}(3)$ and $\mathrm{SE}(3)$, but in $\mathrm{SO}(3)$ What happens to $\mathfrak{so}(3)$ in Lie algebra when two matrix multiplications are completed? Conversely, when $\mathfrak{so}(3)$ is used to add two Lie algebras, does $\mathrm{SO}(3)$ correspond to the product of the two matrices? If established, it is equivalent to:

\[
  \exp \left( {\boldsymbol{\phi} _1^ \wedge } \right)\exp \left( {\boldsymbol{\phi} _2^ \wedge } \right) = \exp \left( {{{\left( {{\boldsymbol{\phi} _1} + {\boldsymbol{\phi} _2}} \right)}^ \wedge }} \right) ?
\]

If $\boldsymbol{\phi}_1, \boldsymbol{\phi}_2$ is a scalar, then obviously this is true; but here we calculate the exponential function of \textbf{matrix} instead of the scalar exponent. In other words, we are studying whether the following formula holds:

\[
  \ln \left( \exp \left( \bm{A} \right) \exp \left( \bm{B} \right) \right) = \bm{A} + \bm{B} \; ?
\]

Unfortunately, this formula does not hold true in the matrix. The complete form of the product of the two Lie algebra indices is represented by the Baker-Campbell-Hausdorff formula (BCH formula)\footnote{ see \url{https://en.wikipedia.org/wiki/Baker-Campbell-Hausdorff\_formula}. } given. Due to the complexity of its complete form, we only give the first few items of its expansion:

\begin{equation}
  \ln \left( {\exp \left( \bm{A} \right)\exp \left( \bm{B} \right)} \right) = \bm{A} + \bm{B} + \frac{1}{2}\left[ {\bm{A}, \bm{B}} \right] + \frac{1}{{12}}\left[ {\bm{A},\left[ {\bm{A}, \bm{B}} \right]} \right] - \frac{1}{{12}}\left[ {\bm{B},\left[ {\bm{A},\bm{B}} \right]} \right] +  \cdots
\end{equation}

Where $[]$ is the brackets. The BCH formula tells us that when dealing with the product of two matrix indices, they produce some remainders consisting of Lie brackets. In particular, consider the Lie algebra on $\mathrm{SO}(3)$$\ln { \left( {\exp \left( { \boldsymbol{\phi} _1^ \wedge } \right)\exp \left ( {\boldsymbol{\phi} _2^ \wedge } \right)} \right) ^ \vee }$, when $\boldsymbol{\phi_1}$ or $\boldsymbol{\phi_2}$ is small, small Items with more than two quantities can be ignored. At this time, BCH has a linear approximation \footnote{ concrete derivation of BCH specific form and approximate expression, this book is not discussed, please refer to the document \cite{Barfoot2016}. }:

\begin{equation}
  \ln { \left( {\exp \left( { \boldsymbol{\phi} _1^ \wedge } \right)\exp \left( {\boldsymbol{\phi} _2^ \wedge } \right)} \right) ^ \vee } \approx \left\{
    \begin{array}{l}
      {\bm{J}_l}{\left( {{\boldsymbol{\phi} _2}} \right)^{ - 1}}{ \boldsymbol{\phi} _1} + {\boldsymbol{\phi} _2} \quad \text{when} \boldsymbol{\phi}_1 \text{Small amount},\\
      {\bm{J}_r}{\left( {{\boldsymbol{\phi} _1}} \right)^{ - 1}}{\boldsymbol{\phi} _2} + {\boldsymbol{\phi} _1} \quad \text{when} \boldsymbol{\phi}_2 \text{Small amount}.
  \end{array} \right.
\end{equation}

Take the first approximation as an example. This formula tells us to multiply a tiny rotation matrix $\bm{R}_1$ to a rotation matrix $\bm{R}_2$ (the Lie algebra is $\boldsymbol{\phi}_2$) (Li algebra is $\boldsymbol{\phi} _1$) can be approximated as the original Lie algebra $\boldsymbol{\phi}_2$Added an item${\bm{J}_l}{\left( {{\boldsymbol{\phi} _2}} \right)^{ - 1}}{ \boldsymbol{\phi} _1}$. 
Similarly, the second approximation describes the case where the right multiplied by a small displacement. Therefore, under the BCH approximation, the Lie algebra is divided into a left-multiplying approximation and a right-multiplying approximation. In use, we must pay attention to whether the left-multiplier model or the right-multiply model is used.

This book takes the left multiplication as an example. Left multiply BCH approximates Jacobian $\bm{J}_l$ is actually the content of the form ~\eqref{eq:lieAlgebraJacobian}~:

\begin{equation} 
{ \bm{J}_l} = \bm{J} = \frac{{\sin \theta }}{\theta } \bm{I} + \left( {1 - \frac{{\sin \theta }}{\theta }} \right) \bm{a} { \bm{a}^\mathrm{T}} + \frac{{1 - \cos \theta }}{\theta }{ \bm{a}^ \wedge}.
\end{equation}

Its inverse is:

\begin{equation}
\bm{J}_l^{ - 1} = \frac{\theta }{2}\cot \frac{\theta }{2} \bm{I} + \left( {1 - \frac{\theta } {2}\cot \frac{\theta }{2}} \right) \bm{a} {\bm{a}^\mathrm{T}} - \frac{\theta }{2}{ \bm{ a}^ \wedge }.
\end{equation}

Right-handed Jacobi only needs to take a negative sign for the argument:

\begin{equation}
\bm{J}_r(\boldsymbol{\phi}) =\bm{J}_l(-\boldsymbol{\phi}) .
\end{equation}

In this way, we can talk about the relationship between Lie group multiplication and Lie algebra addition.

For the convenience of the reader, we restate the meaning of the BCH approximation. Suppose that for a rotation of $\bm{R}$, the corresponding Lie algebra is $\boldsymbol{\phi}$. We give it a small rotation to the left, denoted as $\Delta \bm{R}$, and the corresponding Lie algebra is $\Delta \boldsymbol{\phi}$. Then, on Lie group, the result is $ \Delta \bm{R} \cdot \bm{R}$, and on the Lie algebra, according to the BCH approximation, it is $\bm{J}_l^{-1 } (\boldsymbol{\phi}) \Delta \boldsymbol{\phi} + \boldsymbol{\phi}$. Combined, you can simply write:

\begin{equation}
\exp \left( {\Delta { \boldsymbol{\phi} ^ \wedge }} \right)\exp \left( {{ \boldsymbol{\phi} ^ \wedge }} \right) = \exp \left( {{{\left( { \boldsymbol{\phi} + \bm{J}_l^{ - 1}\left( \boldsymbol{\phi} \right)\Delta \boldsymbol{\phi} } \right)} ^ \wedge }} \right).
\end{equation}

Conversely, if we add on Lie algebra and add $\boldsymbol{\phi}$ to $\Delta \boldsymbol{\phi}$, we can approximate the multiplication of the left and right Jacobian on the Lie group:

\begin{equation}
\exp \left( {{{\left( { \boldsymbol{\phi}  + \Delta \boldsymbol{\phi} } \right)}^ \wedge }} \right) = \exp \left( {{{\left( {{ \bm{J}_l}\Delta \boldsymbol{\phi} } \right)}^ \wedge }} \right)\exp \left( {{ \boldsymbol{\phi} ^ \wedge }} \right) = \exp \left( {{\boldsymbol{\phi} ^ \wedge }} \right)\exp \left( {{{\left( {{\bm{J}_r}\Delta \boldsymbol{\phi} } \right)}^ \wedge }} \right).
\end{equation}

This provides a theoretical basis for calculus after Lie algebra. Similarly, for $\mathrm{SE}(3)$, there is a similar BCH approximation:

\begin{equation}
\exp \left( {\Delta {\boldsymbol{\xi} ^ \wedge }} \right)\exp \left( {{ \boldsymbol{\xi} ^ \wedge }} \right) \approx \exp \left ( {{{\left( {{ \bm{\mathcal{J}}_l^{-1} }\Delta \boldsymbol{\xi} + \boldsymbol{\xi} } \right)}^ \wedge }} \right),
\end{equation}

\begin{equation}
\exp \left( {{ \boldsymbol{\xi} ^ \wedge }} \right) \exp \left( {\Delta {\boldsymbol{\xi} ^ \wedge }} \right) \approx \exp \left ( {{{\left( {{ \bm{\mathcal{J}}_r^{-1} }\Delta \boldsymbol{\xi} + \boldsymbol{\xi} } \right)}^ \wedge }} \right).
\end{equation}

Here the $\bm{\mathcal{J}}_l$ form is more complicated. It is a matrix of $6 \times 6$. Readers can refer to the contents of the \cite{Barfoot2016} formulas (7.82) and (7.83). Since we did not use the Jacobi in the calculation, the actual form is omitted here.


\subsubsection{Install OpenCV}

OpenCV\footnote{Official homepage:\url{http://opencv.org}. } provides a large number of open source image algorithms, is a library of image processing algorithms used in computer vision. This book also uses OpenCV for basic image processing. Before using it, it is recommended that the reader install it from the source code. Under Ubuntu, there are two ways to \textbf{install from source code} and \textbf{install library file only}:

\begin{enumerate}
	\item Installation from source code means downloading all OpenCV source code from the OpenCV website and compiling and installing it on the machine for use. The advantage is that the version you can choose is rich, and you can see the source code, but it takes some compilation time.
	\item only installs the library file, which means to install the library file compiled by Ubuntu community personnel through Ubuntu, so there is no need to recompile it.
\end{enumerate}

Since we are using a newer version of OpenCV, we must install it from the source code. As a result, you can adjust some compile options to match the programming environment (for example, GPU acceleration is not required); in addition, source code installation can use some additional features. OpenCV currently maintains two major versions, divided into OpenCV 2.4 series and OpenCV 3 series. This book uses the OpenCV \textbf{3} series.

Since the OpenCV project is relatively large, it is not placed under the 3rdparty of this book. Readers can download from ~\url{http://opencv.org/downloads.html} and choose OpenCV for Linux. You will get a tarball like opencv-3.1.0.zip. Extract it to any directory, we found that OpenCV is also a cmake project.

Before compiling, first install the OpenCV dependencies:

\begin{lstlisting}[language=sh,caption=terminal input:]
Sudo apt-get install build-essential libgtk2.0-dev libvtk5-dev libjpeg-dev libtiff4-dev libjasper-dev libopenexr-dev libtbb-dev
\end{lstlisting}

In fact, OpenCV has a lot of dependencies, and the lack of some compiled items will affect some of its features (but we won't use all the features). OpenCV will check if the dependencies will be installed and adjust its functionality during the cmake phase. If you have a GPU on your computer and have related dependencies installed, OpenCV will also speed up the GPU. But for this book, the above dependencies are enough.

Subsequent compilation and installation is the same as the normal cmake project. After make, call sudo make install to install OpenCV on your machine (instead of just compiling it). Depending on the machine configuration, this compilation process can take anywhere from twenty minutes to an hour. If your CPU is strong, you can use a command like "make -j4" to call multiple threads to compile (the arguments after -j are the number of threads used). After installation, OpenCV is stored by default in the /usr/local directory. You can find the location where OpenCV header files and library files are installed and see where they are. In addition, if you have previously installed the OpenCV 2 series, it is recommended that you install OpenCV 3 somewhere else (think how this should work).



\section{Practice: Eigen Geometry Module}

\subsection{BCH formula and approximation}

A major motivation for using Lie algebra is to optimize, and the derivative is a very necessary information in the optimization process (we will cover it in detail in Lecture 6). Let's consider a problem below. Although we have already understood the relationship between Lie group and Lie algebra on $\mathrm{SO}(3)$ and $\mathrm{SE}(3)$, but in $\mathrm{SO}(3)$ What happens to $\mathfrak{so}(3)$ in Lie algebra when two matrix multiplications are completed? Conversely, when $\mathfrak{so}(3)$ is used to add two Lie algebras, does $\mathrm{SO}(3)$ correspond to the product of the two matrices? If established, it is equivalent to:

\[
  \exp \left( {\boldsymbol{\phi} _1^ \wedge } \right)\exp \left( {\boldsymbol{\phi} _2^ \wedge } \right) = \exp \left( {{{\left( {{\boldsymbol{\phi} _1} + {\boldsymbol{\phi} _2}} \right)}^ \wedge }} \right) ?
\]

If $\boldsymbol{\phi}_1, \boldsymbol{\phi}_2$ is a scalar, then obviously this is true; but here we calculate the exponential function of \textbf{matrix} instead of the scalar exponent. In other words, we are studying whether the following formula holds:

\[
  \ln \left( \exp \left( \bm{A} \right) \exp \left( \bm{B} \right) \right) = \bm{A} + \bm{B} \; ?
\]

Unfortunately, this formula does not hold true in the matrix. The complete form of the product of the two Lie algebra indices is represented by the Baker-Campbell-Hausdorff formula (BCH formula)\footnote{ see \url{https://en.wikipedia.org/wiki/Baker-Campbell-Hausdorff\_formula}. } given. Due to the complexity of its complete form, we only give the first few items of its expansion:

\begin{equation}
  \ln \left( {\exp \left( \bm{A} \right)\exp \left( \bm{B} \right)} \right) = \bm{A} + \bm{B} + \frac{1}{2}\left[ {\bm{A}, \bm{B}} \right] + \frac{1}{{12}}\left[ {\bm{A},\left[ {\bm{A}, \bm{B}} \right]} \right] - \frac{1}{{12}}\left[ {\bm{B},\left[ {\bm{A},\bm{B}} \right]} \right] +  \cdots
\end{equation}

Where $[]$ is the brackets. The BCH formula tells us that when dealing with the product of two matrix indices, they produce some remainders consisting of Lie brackets. In particular, consider the Lie algebra on $\mathrm{SO}(3)$$\ln { \left( {\exp \left( { \boldsymbol{\phi} _1^ \wedge } \right)\exp \left ( {\boldsymbol{\phi} _2^ \wedge } \right)} \right) ^ \vee }$, when $\boldsymbol{\phi_1}$ or $\boldsymbol{\phi_2}$ is small, small Items with more than two quantities can be ignored. At this time, BCH has a linear approximation \footnote{ concrete derivation of BCH specific form and approximate expression, this book is not discussed, please refer to the document \cite{Barfoot2016}. }:

\begin{equation}
  \ln { \left( {\exp \left( { \boldsymbol{\phi} _1^ \wedge } \right)\exp \left( {\boldsymbol{\phi} _2^ \wedge } \right)} \right) ^ \vee } \approx \left\{
    \begin{array}{l}
      {\bm{J}_l}{\left( {{\boldsymbol{\phi} _2}} \right)^{ - 1}}{ \boldsymbol{\phi} _1} + {\boldsymbol{\phi} _2} \quad \text{when} \boldsymbol{\phi}_1 \text{Small amount},\\
      {\bm{J}_r}{\left( {{\boldsymbol{\phi} _1}} \right)^{ - 1}}{\boldsymbol{\phi} _2} + {\boldsymbol{\phi} _1} \quad \text{when} \boldsymbol{\phi}_2 \text{Small amount}.
  \end{array} \right.
\end{equation}

Take the first approximation as an example. This formula tells us to multiply a tiny rotation matrix $\bm{R}_1$ to a rotation matrix $\bm{R}_2$ (the Lie algebra is $\boldsymbol{\phi}_2$) (Li algebra is $\boldsymbol{\phi} _1$) can be approximated as the original Lie algebra $\boldsymbol{\phi}_2$Added an item${\bm{J}_l}{\left( {{\boldsymbol{\phi} _2}} \right)^{ - 1}}{ \boldsymbol{\phi} _1}$. 
Similarly, the second approximation describes the case where the right multiplied by a small displacement. Therefore, under the BCH approximation, the Lie algebra is divided into a left-multiplying approximation and a right-multiplying approximation. In use, we must pay attention to whether the left-multiplier model or the right-multiply model is used.

This book takes the left multiplication as an example. Left multiply BCH approximates Jacobian $\bm{J}_l$ is actually the content of the form ~\eqref{eq:lieAlgebraJacobian}~:

\begin{equation} 
{ \bm{J}_l} = \bm{J} = \frac{{\sin \theta }}{\theta } \bm{I} + \left( {1 - \frac{{\sin \theta }}{\theta }} \right) \bm{a} { \bm{a}^\mathrm{T}} + \frac{{1 - \cos \theta }}{\theta }{ \bm{a}^ \wedge}.
\end{equation}

Its inverse is:

\begin{equation}
\bm{J}_l^{ - 1} = \frac{\theta }{2}\cot \frac{\theta }{2} \bm{I} + \left( {1 - \frac{\theta } {2}\cot \frac{\theta }{2}} \right) \bm{a} {\bm{a}^\mathrm{T}} - \frac{\theta }{2}{ \bm{ a}^ \wedge }.
\end{equation}

Right-handed Jacobi only needs to take a negative sign for the argument:

\begin{equation}
\bm{J}_r(\boldsymbol{\phi}) =\bm{J}_l(-\boldsymbol{\phi}) .
\end{equation}

In this way, we can talk about the relationship between Lie group multiplication and Lie algebra addition.

For the convenience of the reader, we restate the meaning of the BCH approximation. Suppose that for a rotation of $\bm{R}$, the corresponding Lie algebra is $\boldsymbol{\phi}$. We give it a small rotation to the left, denoted as $\Delta \bm{R}$, and the corresponding Lie algebra is $\Delta \boldsymbol{\phi}$. Then, on Lie group, the result is $ \Delta \bm{R} \cdot \bm{R}$, and on the Lie algebra, according to the BCH approximation, it is $\bm{J}_l^{-1 } (\boldsymbol{\phi}) \Delta \boldsymbol{\phi} + \boldsymbol{\phi}$. Combined, you can simply write:

\begin{equation}
\exp \left( {\Delta { \boldsymbol{\phi} ^ \wedge }} \right)\exp \left( {{ \boldsymbol{\phi} ^ \wedge }} \right) = \exp \left( {{{\left( { \boldsymbol{\phi} + \bm{J}_l^{ - 1}\left( \boldsymbol{\phi} \right)\Delta \boldsymbol{\phi} } \right)} ^ \wedge }} \right).
\end{equation}

Conversely, if we add on Lie algebra and add $\boldsymbol{\phi}$ to $\Delta \boldsymbol{\phi}$, we can approximate the multiplication of the left and right Jacobian on the Lie group:

\begin{equation}
\exp \left( {{{\left( { \boldsymbol{\phi}  + \Delta \boldsymbol{\phi} } \right)}^ \wedge }} \right) = \exp \left( {{{\left( {{ \bm{J}_l}\Delta \boldsymbol{\phi} } \right)}^ \wedge }} \right)\exp \left( {{ \boldsymbol{\phi} ^ \wedge }} \right) = \exp \left( {{\boldsymbol{\phi} ^ \wedge }} \right)\exp \left( {{{\left( {{\bm{J}_r}\Delta \boldsymbol{\phi} } \right)}^ \wedge }} \right).
\end{equation}

This provides a theoretical basis for calculus after Lie algebra. Similarly, for $\mathrm{SE}(3)$, there is a similar BCH approximation:

\begin{equation}
\exp \left( {\Delta {\boldsymbol{\xi} ^ \wedge }} \right)\exp \left( {{ \boldsymbol{\xi} ^ \wedge }} \right) \approx \exp \left ( {{{\left( {{ \bm{\mathcal{J}}_l^{-1} }\Delta \boldsymbol{\xi} + \boldsymbol{\xi} } \right)}^ \wedge }} \right),
\end{equation}

\begin{equation}
\exp \left( {{ \boldsymbol{\xi} ^ \wedge }} \right) \exp \left( {\Delta {\boldsymbol{\xi} ^ \wedge }} \right) \approx \exp \left ( {{{\left( {{ \bm{\mathcal{J}}_r^{-1} }\Delta \boldsymbol{\xi} + \boldsymbol{\xi} } \right)}^ \wedge }} \right).
\end{equation}

Here the $\bm{\mathcal{J}}_l$ form is more complicated. It is a matrix of $6 \times 6$. Readers can refer to the contents of the \cite{Barfoot2016} formulas (7.82) and (7.83). Since we did not use the Jacobi in the calculation, the actual form is omitted here.


\subsubsection{Install OpenCV}

OpenCV\footnote{Official homepage:\url{http://opencv.org}. } provides a large number of open source image algorithms, is a library of image processing algorithms used in computer vision. This book also uses OpenCV for basic image processing. Before using it, it is recommended that the reader install it from the source code. Under Ubuntu, there are two ways to \textbf{install from source code} and \textbf{install library file only}:

\begin{enumerate}
	\item Installation from source code means downloading all OpenCV source code from the OpenCV website and compiling and installing it on the machine for use. The advantage is that the version you can choose is rich, and you can see the source code, but it takes some compilation time.
	\item only installs the library file, which means to install the library file compiled by Ubuntu community personnel through Ubuntu, so there is no need to recompile it.
\end{enumerate}

Since we are using a newer version of OpenCV, we must install it from the source code. As a result, you can adjust some compile options to match the programming environment (for example, GPU acceleration is not required); in addition, source code installation can use some additional features. OpenCV currently maintains two major versions, divided into OpenCV 2.4 series and OpenCV 3 series. This book uses the OpenCV \textbf{3} series.

Since the OpenCV project is relatively large, it is not placed under the 3rdparty of this book. Readers can download from ~\url{http://opencv.org/downloads.html} and choose OpenCV for Linux. You will get a tarball like opencv-3.1.0.zip. Extract it to any directory, we found that OpenCV is also a cmake project.

Before compiling, first install the OpenCV dependencies:

\begin{lstlisting}[language=sh,caption=terminal input:]
Sudo apt-get install build-essential libgtk2.0-dev libvtk5-dev libjpeg-dev libtiff4-dev libjasper-dev libopenexr-dev libtbb-dev
\end{lstlisting}

In fact, OpenCV has a lot of dependencies, and the lack of some compiled items will affect some of its features (but we won't use all the features). OpenCV will check if the dependencies will be installed and adjust its functionality during the cmake phase. If you have a GPU on your computer and have related dependencies installed, OpenCV will also speed up the GPU. But for this book, the above dependencies are enough.

Subsequent compilation and installation is the same as the normal cmake project. After make, call sudo make install to install OpenCV on your machine (instead of just compiling it). Depending on the machine configuration, this compilation process can take anywhere from twenty minutes to an hour. If your CPU is strong, you can use a command like "make -j4" to call multiple threads to compile (the arguments after -j are the number of threads used). After installation, OpenCV is stored by default in the /usr/local directory. You can find the location where OpenCV header files and library files are installed and see where they are. In addition, if you have previously installed the OpenCV 2 series, it is recommended that you install OpenCV 3 somewhere else (think how this should work).



\section{Practice: Eigen Geometry Module}

\subsection{BCH formula and approximation}

A major motivation for using Lie algebra is to optimize, and the derivative is a very necessary information in the optimization process (we will cover it in detail in Lecture 6). Let's consider a problem below. Although we have already understood the relationship between Lie group and Lie algebra on $\mathrm{SO}(3)$ and $\mathrm{SE}(3)$, but in $\mathrm{SO}(3)$ What happens to $\mathfrak{so}(3)$ in Lie algebra when two matrix multiplications are completed? Conversely, when $\mathfrak{so}(3)$ is used to add two Lie algebras, does $\mathrm{SO}(3)$ correspond to the product of the two matrices? If established, it is equivalent to:

\[
  \exp \left( {\boldsymbol{\phi} _1^ \wedge } \right)\exp \left( {\boldsymbol{\phi} _2^ \wedge } \right) = \exp \left( {{{\left( {{\boldsymbol{\phi} _1} + {\boldsymbol{\phi} _2}} \right)}^ \wedge }} \right) ?
\]

If $\boldsymbol{\phi}_1, \boldsymbol{\phi}_2$ is a scalar, then obviously this is true; but here we calculate the exponential function of \textbf{matrix} instead of the scalar exponent. In other words, we are studying whether the following formula holds:

\[
  \ln \left( \exp \left( \bm{A} \right) \exp \left( \bm{B} \right) \right) = \bm{A} + \bm{B} \; ?
\]

Unfortunately, this formula does not hold true in the matrix. The complete form of the product of the two Lie algebra indices is represented by the Baker-Campbell-Hausdorff formula (BCH formula)\footnote{ see \url{https://en.wikipedia.org/wiki/Baker-Campbell-Hausdorff\_formula}. } given. Due to the complexity of its complete form, we only give the first few items of its expansion:

\begin{equation}
  \ln \left( {\exp \left( \bm{A} \right)\exp \left( \bm{B} \right)} \right) = \bm{A} + \bm{B} + \frac{1}{2}\left[ {\bm{A}, \bm{B}} \right] + \frac{1}{{12}}\left[ {\bm{A},\left[ {\bm{A}, \bm{B}} \right]} \right] - \frac{1}{{12}}\left[ {\bm{B},\left[ {\bm{A},\bm{B}} \right]} \right] +  \cdots
\end{equation}

Where $[]$ is the brackets. The BCH formula tells us that when dealing with the product of two matrix indices, they produce some remainders consisting of Lie brackets. In particular, consider the Lie algebra on $\mathrm{SO}(3)$$\ln { \left( {\exp \left( { \boldsymbol{\phi} _1^ \wedge } \right)\exp \left ( {\boldsymbol{\phi} _2^ \wedge } \right)} \right) ^ \vee }$, when $\boldsymbol{\phi_1}$ or $\boldsymbol{\phi_2}$ is small, small Items with more than two quantities can be ignored. At this time, BCH has a linear approximation \footnote{ concrete derivation of BCH specific form and approximate expression, this book is not discussed, please refer to the document \cite{Barfoot2016}. }:

\begin{equation}
  \ln { \left( {\exp \left( { \boldsymbol{\phi} _1^ \wedge } \right)\exp \left( {\boldsymbol{\phi} _2^ \wedge } \right)} \right) ^ \vee } \approx \left\{
    \begin{array}{l}
      {\bm{J}_l}{\left( {{\boldsymbol{\phi} _2}} \right)^{ - 1}}{ \boldsymbol{\phi} _1} + {\boldsymbol{\phi} _2} \quad \text{when} \boldsymbol{\phi}_1 \text{Small amount},\\
      {\bm{J}_r}{\left( {{\boldsymbol{\phi} _1}} \right)^{ - 1}}{\boldsymbol{\phi} _2} + {\boldsymbol{\phi} _1} \quad \text{when} \boldsymbol{\phi}_2 \text{Small amount}.
  \end{array} \right.
\end{equation}

Take the first approximation as an example. This formula tells us to multiply a tiny rotation matrix $\bm{R}_1$ to a rotation matrix $\bm{R}_2$ (the Lie algebra is $\boldsymbol{\phi}_2$) (Li algebra is $\boldsymbol{\phi} _1$) can be approximated as the original Lie algebra $\boldsymbol{\phi}_2$Added an item${\bm{J}_l}{\left( {{\boldsymbol{\phi} _2}} \right)^{ - 1}}{ \boldsymbol{\phi} _1}$. 
Similarly, the second approximation describes the case where the right multiplied by a small displacement. Therefore, under the BCH approximation, the Lie algebra is divided into a left-multiplying approximation and a right-multiplying approximation. In use, we must pay attention to whether the left-multiplier model or the right-multiply model is used.

This book takes the left multiplication as an example. Left multiply BCH approximates Jacobian $\bm{J}_l$ is actually the content of the form ~\eqref{eq:lieAlgebraJacobian}~:

\begin{equation} 
{ \bm{J}_l} = \bm{J} = \frac{{\sin \theta }}{\theta } \bm{I} + \left( {1 - \frac{{\sin \theta }}{\theta }} \right) \bm{a} { \bm{a}^\mathrm{T}} + \frac{{1 - \cos \theta }}{\theta }{ \bm{a}^ \wedge}.
\end{equation}

Its inverse is:

\begin{equation}
\bm{J}_l^{ - 1} = \frac{\theta }{2}\cot \frac{\theta }{2} \bm{I} + \left( {1 - \frac{\theta } {2}\cot \frac{\theta }{2}} \right) \bm{a} {\bm{a}^\mathrm{T}} - \frac{\theta }{2}{ \bm{ a}^ \wedge }.
\end{equation}

Right-handed Jacobi only needs to take a negative sign for the argument:

\begin{equation}
\bm{J}_r(\boldsymbol{\phi}) =\bm{J}_l(-\boldsymbol{\phi}) .
\end{equation}

In this way, we can talk about the relationship between Lie group multiplication and Lie algebra addition.

For the convenience of the reader, we restate the meaning of the BCH approximation. Suppose that for a rotation of $\bm{R}$, the corresponding Lie algebra is $\boldsymbol{\phi}$. We give it a small rotation to the left, denoted as $\Delta \bm{R}$, and the corresponding Lie algebra is $\Delta \boldsymbol{\phi}$. Then, on Lie group, the result is $ \Delta \bm{R} \cdot \bm{R}$, and on the Lie algebra, according to the BCH approximation, it is $\bm{J}_l^{-1 } (\boldsymbol{\phi}) \Delta \boldsymbol{\phi} + \boldsymbol{\phi}$. Combined, you can simply write:

\begin{equation}
\exp \left( {\Delta { \boldsymbol{\phi} ^ \wedge }} \right)\exp \left( {{ \boldsymbol{\phi} ^ \wedge }} \right) = \exp \left( {{{\left( { \boldsymbol{\phi} + \bm{J}_l^{ - 1}\left( \boldsymbol{\phi} \right)\Delta \boldsymbol{\phi} } \right)} ^ \wedge }} \right).
\end{equation}

Conversely, if we add on Lie algebra and add $\boldsymbol{\phi}$ to $\Delta \boldsymbol{\phi}$, we can approximate the multiplication of the left and right Jacobian on the Lie group:

\begin{equation}
\exp \left( {{{\left( { \boldsymbol{\phi}  + \Delta \boldsymbol{\phi} } \right)}^ \wedge }} \right) = \exp \left( {{{\left( {{ \bm{J}_l}\Delta \boldsymbol{\phi} } \right)}^ \wedge }} \right)\exp \left( {{ \boldsymbol{\phi} ^ \wedge }} \right) = \exp \left( {{\boldsymbol{\phi} ^ \wedge }} \right)\exp \left( {{{\left( {{\bm{J}_r}\Delta \boldsymbol{\phi} } \right)}^ \wedge }} \right).
\end{equation}

This provides a theoretical basis for calculus after Lie algebra. Similarly, for $\mathrm{SE}(3)$, there is a similar BCH approximation:

\begin{equation}
\exp \left( {\Delta {\boldsymbol{\xi} ^ \wedge }} \right)\exp \left( {{ \boldsymbol{\xi} ^ \wedge }} \right) \approx \exp \left ( {{{\left( {{ \bm{\mathcal{J}}_l^{-1} }\Delta \boldsymbol{\xi} + \boldsymbol{\xi} } \right)}^ \wedge }} \right),
\end{equation}

\begin{equation}
\exp \left( {{ \boldsymbol{\xi} ^ \wedge }} \right) \exp \left( {\Delta {\boldsymbol{\xi} ^ \wedge }} \right) \approx \exp \left ( {{{\left( {{ \bm{\mathcal{J}}_r^{-1} }\Delta \boldsymbol{\xi} + \boldsymbol{\xi} } \right)}^ \wedge }} \right).
\end{equation}

Here the $\bm{\mathcal{J}}_l$ form is more complicated. It is a matrix of $6 \times 6$. Readers can refer to the contents of the \cite{Barfoot2016} formulas (7.82) and (7.83). Since we did not use the Jacobi in the calculation, the actual form is omitted here.


\subsubsection{Install OpenCV}

OpenCV\footnote{Official homepage:\url{http://opencv.org}. } provides a large number of open source image algorithms, is a library of image processing algorithms used in computer vision. This book also uses OpenCV for basic image processing. Before using it, it is recommended that the reader install it from the source code. Under Ubuntu, there are two ways to \textbf{install from source code} and \textbf{install library file only}:

\begin{enumerate}
	\item Installation from source code means downloading all OpenCV source code from the OpenCV website and compiling and installing it on the machine for use. The advantage is that the version you can choose is rich, and you can see the source code, but it takes some compilation time.
	\item only installs the library file, which means to install the library file compiled by Ubuntu community personnel through Ubuntu, so there is no need to recompile it.
\end{enumerate}

Since we are using a newer version of OpenCV, we must install it from the source code. As a result, you can adjust some compile options to match the programming environment (for example, GPU acceleration is not required); in addition, source code installation can use some additional features. OpenCV currently maintains two major versions, divided into OpenCV 2.4 series and OpenCV 3 series. This book uses the OpenCV \textbf{3} series.

Since the OpenCV project is relatively large, it is not placed under the 3rdparty of this book. Readers can download from ~\url{http://opencv.org/downloads.html} and choose OpenCV for Linux. You will get a tarball like opencv-3.1.0.zip. Extract it to any directory, we found that OpenCV is also a cmake project.

Before compiling, first install the OpenCV dependencies:

\begin{lstlisting}[language=sh,caption=terminal input:]
Sudo apt-get install build-essential libgtk2.0-dev libvtk5-dev libjpeg-dev libtiff4-dev libjasper-dev libopenexr-dev libtbb-dev
\end{lstlisting}

In fact, OpenCV has a lot of dependencies, and the lack of some compiled items will affect some of its features (but we won't use all the features). OpenCV will check if the dependencies will be installed and adjust its functionality during the cmake phase. If you have a GPU on your computer and have related dependencies installed, OpenCV will also speed up the GPU. But for this book, the above dependencies are enough.

Subsequent compilation and installation is the same as the normal cmake project. After make, call sudo make install to install OpenCV on your machine (instead of just compiling it). Depending on the machine configuration, this compilation process can take anywhere from twenty minutes to an hour. If your CPU is strong, you can use a command like "make -j4" to call multiple threads to compile (the arguments after -j are the number of threads used). After installation, OpenCV is stored by default in the /usr/local directory. You can find the location where OpenCV header files and library files are installed and see where they are. In addition, if you have previously installed the OpenCV 2 series, it is recommended that you install OpenCV 3 somewhere else (think how this should work).


